\documentclass{article}
\title{Doremi}
\author{Jason R. Fruit}
\date{2016}

\usepackage{lineno}

\begin{document}

\maketitle

\abstract{\textbf{Doremi} is a simple music-representation designed
  especially for simple multi-part works in a vocal style, e.g. hymn
  tunes and part songs.  It is based on movable-\textit{do} solfège,
  and is intended to require as little understanding of traditional
  musical notation as possible.}

\tableofcontents

\section{A simple example}

Here is the common hymn tune \textit{Old Hundred} written in Doremi:

\linenumbers
\begin{verbatim}
title: "Old Hundred"
composer: "Louis Bourgeois"
key: A major
time: 4/2
partial: 2
voices: [{name: soprano
          octave: 1
          content: [2 do do - ti la sol + do re fermata mi
                    mi mi mi re do fa mi fermata re
                    do re mi re do - la ti + fermata do
                    sol mi do re fa mi re fermata do]}
         {name: alto
	  octave: 0
	  content: [2 sol sol sol mi mi mi sol sol
	  	    sol sol + do - ti + do do do - ti
		    + do - ti + do - sol sol la sol sol
		    + do - sol la ti 4 slur la ti + 2 do - ti + do]}
         {name: tenor
	  octave: 0
	  content: [2 mi mi re do - ti la ti + do
	  	    do mi sol sol mi la sol sol
		    mi sol sol fa sol 4 slur fa mi 2 re mi
		    mi do mi sol la sol fa mi]}
         {name: bass
	  octave: 0
	  content: [2 do do - sol la mi la sol do
	  	    + do do do - sol la fa do sol
	  	    la sol + do - sol mi fa sol do
	  	    + do do - la sol re 4 slur mi fa 2 sol do]}]
\end{verbatim}
\nolinenumbers

\section{The header}

The heading (lines 1-5) is standard, and all its lines are required
except for line 5, beginning \texttt{partial}.  The title and composer
(lines 1 and 2) are mostly self-explanatory; note that they
\textit{must} be enclosed in double quotation marks.

The key signature (line 3) is the name of the key note (sharp is
indicated by \texttt{\#}, flat by \texttt{b}) followed by its quality,
either major or minor, e.g. \texttt{Ab major}, \texttt{F\# minor}.

The time signature (line 4) must be written as a fraction
(abbreviations such as \texttt{C} will not work).  There is no
provision for mixed meter signatures.

Line 5, beginning \texttt{partial:}, is optional; it specifies the
length of the first measure if it is a partial (as opposed to
complete) measure; the possible values are those of note durations
(see \textbf{Note modifiers}, below).

\section{Voices}

A tune also has voices, which are indicated by the tag
\texttt{voices:}.  It can have as many or as few voices as are
specified by the template for which it is intended.  The list of
voices is enclosed in square brackets (\texttt{{[]}}), and each voice
is enclosed in curly braces (\texttt{\{\}}).

Each voice has the same structure, similar to that of a tune.  It has
a name (as in line 6); a beginning octave specification (line 7), and
a list of notes (lines 8-11).

\subsection{Name}

The name is mostly self-explanatory, except that it is not a string
and cannot contain spaces.  It must begin with an alphabetic
character, and can only contain letters, numbers, and hyphens.  It is
used in conjunction with templates to place the voices correctly.

\subsection{Octave}

The octave is specified by a positive or negative integer.  0 is the
octave nearest the center, roughly speaking, of the average female
voice; each higher octave adds 1, each lower octave subtracts.  (The
octave can be modified by an overall octave offset specified at the
command line, but the relation between voices remains constant.)

\subsection{Content}

The content of a voice consists of a list of notes and their
modifiers.

\subsection{Notes}

The notes are specified by solfège syllables and a subset of their
chromatic alterations:\\

\begin{tabular}{| c | c |}

  \hline
  \textbf{Syllable} & \textbf{Pitch} \\
  \hline
  do & do\\
  di & do$\sharp$\\
  ra & re$\flat$\\
  re & re\\
  ri & re$\sharp$\\
  me & mi$\flat$\\
  mi & mi\\
  fa & fa\\
  fi & fa$\sharp$\\
  se & sol$\flat$\\
  sol & sol\\
  si & sol$\sharp$\\
  le & la$\flat$\\
  la & la\\
  li & la$\sharp$\\
  te & ti$\flat$\\
  ti & ti\\
  \hline
  
\end{tabular}

\subsection{Note modifiers}

Each note may be preceded by a number of modifiers, which may be a
number specifying a new note duration, a time signature, or one of the
following:\\

\begin{tabular}{| c | c |}

  \hline
  \textbf{Modifier} & \textbf{Meaning}\\
  \hline
  \texttt{fermata} & Add a fermata to the next note\\
  \texttt{slur} & Start (or continue) a slur at the next note\\
  \texttt{tie} & Tie the next note to the one after\\
  \texttt{-} & Lower the octave of the next note\\
  \texttt{+} & Raise the octave of the next note\\
  \hline

\end{tabular}

\subsubsection*{Durations}

Note durations are specified by a number indicating the denominator of
the duration's fraction of a whole note:\\

\begin{tabular}{| c | c |}

  \hline
  \textbf{Duration} & \textbf{Note type}\\
  \hline
  \texttt{1} & whole\\
  \texttt{2} & half\\
  \texttt{4} & quarter\\
  \texttt{8} & eighth\\
  \texttt{16} & sixteenth\\
  \hline

\end{tabular}\\

\noindent
A duration may be followed by a single dot if the intended note value
is dotted.

\subsubsection*{Time signature}

Time signatures are specified exactly as in the tune header, and need
only be specified in one voice; the other voices will receive the time
signature change implicitly and be barred accordingly.

\end{document}
